\documentclass[a4paper,11pt]{article}

\usepackage[T1]{fontenc}
\usepackage[utf8]{inputenc}
\usepackage{graphicx}
\usepackage{xcolor}

\renewcommand\familydefault{\sfdefault}
\usepackage{tgheros}
\usepackage[defaultmono]{droidmono}

\usepackage{amsmath,amssymb,amsthm,textcomp}
\usepackage{enumerate}
\usepackage{multicol}
\usepackage{tikz}

\usepackage{geometry}
\geometry{total={210mm,297mm},
left=25mm,right=25mm,
bindingoffset=0mm, top=20mm,bottom=20mm}


\linespread{1.3}

\newcommand{\linia}{\rule{\linewidth}{0.5pt}}

\newtheoremstyle{mytheor}
    {1ex}{1ex}{\normalfont}{0pt}{\scshape}{.}{1ex}
    {{\thmname{#1 }}{\thmnumber{#2}}{\thmnote{ (#3)}}}

\theoremstyle{mytheor}
\newtheorem{defi}{Definition}

\makeatletter
\renewcommand{\maketitle}{
\begin{center}
\vspace{2ex}
{\huge \textsc{\@title}}
\vspace{1ex}
\\
\linia\\
\@author \hfill \@date
\vspace{4ex}
\end{center}
}
\makeatother


\usepackage{fancyhdr}
\pagestyle{fancy}
\lhead{}
\chead{}
\rhead{}
\lfoot{Analyzátor klíčových slov}
\cfoot{}
\rfoot{Stránka \thepage}
\renewcommand{\headrulewidth}{0pt}
\renewcommand{\footrulewidth}{0pt}


\usepackage{listings}
\lstset{
    language=perl,
    basicstyle=\ttfamily\small,
    aboveskip={1.0\baselineskip},
    belowskip={1.0\baselineskip},
    columns=fixed,
    extendedchars=true,
    breaklines=true,
    tabsize=4,
    prebreak=\raisebox{0ex}[0ex][0ex]{\ensuremath{\hookleftarrow}},
    frame=lines,
    showtabs=false,
    showspaces=false,
    showstringspaces=false,
    keywordstyle=\color[rgb]{0.627,0.126,0.941},
    commentstyle=\color[rgb]{0.133,0.545,0.133},
    stringstyle=\color[rgb]{01,0,0},
    numbers=left,
    numberstyle=\small,
    stepnumber=1,
    numbersep=10pt,
    captionpos=t,
    escapeinside={\%*}{*)}
}


\begin{document}

\title{Analyzátor klíčových slov - Popis řešení}

\author{Bc. Martin Ševčík, Masarykova univerzita}

\date{28/02/2019}

\maketitle

\section*{Definice úlohy}

Cílem zadání je vytvořit skript v jazyce Perl, který z
\textbf{textového souboru} předaného na standardní vstup
\textbf{vyextrahuje klíčová slova} a tato \textbf{předá na
standardní výstup}, vždy \textbf{jedno klíčové slovo} na
\textbf{jednom řádku}. Cílem je vyextrahovat z dokumentu klíčová
slova vhodná \textbf{pro hledání podobných dokumentů}. Klíčová slova
je třeba na výstupu \textbf{seřadit dle důležitosti}. Řešení lze
omezit pouze na \textbf{české} texty a na jednoslovní klíčová slova.
Při řešení lze použít \textbf{perlové moduly z CPANu}. Vše ostatní
musí být součástí řešení. Váš skript bude kontrolován na
\textbf{linuxovém počítači.}

\section*{Analýza problému a rozbor zadání}

\begin{itemize}
\item Vstup ve formě textového dokumentu text.txt – vstupní text
bude česky, je nutné dodržet kódování utf-8, aby se text správně
načetl.
\item Extrakce klíčových slov určených pro další vyhledávání
podobných textů – definujeme klíčové slovo jako slovo, které je
významově v souvislosti s textem konkrétní (není významově obecné) a
alespoň částečně vystihuje konkrétní myšlenku textu jako celku.
\item Klíčová slova předaná na výstup budou v češtině, každé na
novém řádku a budou seřazena podle důležitosti – musí být dodrženo
kódování utf-8, je třeba vymyslet, jak vypsat každé slovo na nový
řádek, ale ještě před výpisem seřadit vybraná slova dle důležitosti,
kterou je nutné definovat. Výpis slov je také potřeba uložit do nově
vytvořeného textového souboru klicovaslova.txt, který by mohl pro
případné hledání podobných dokumentů sloužit jako zdroj podobnosti.
\item Celý skript musí být kompatibilní s os linux a lze pro jeho
vytvoření využít vhodné CPAN moduly, které pravděpodobně mohou
uvedení celého skriptu do provozu výrazně usnadnit.
\end{itemize}

\section*{Návrh řešení}

\begin{itemize}
\item V celém skriptu je nutné povolit kódování UTF-8, které je
třeba aplikovat i při načítání textu ze zdrojového souboru pomocí
parametru ":encoding(utf8)".
\item Je nutné sestavit samostatnou databázi nebo seznam stop slov,
která jsou příliš obecná a budou z textu před určením klíčových slov
pomocí regulárních výrazů vyfiltrována. 
\item V tomto konkrétním případě bude nejschůdnějším řešením určit
důležitost klíčového slova podle toho, kolikrát se ve vyfiltrovaném
textu vyskytuje. K řešení tohoto problému se nabízí použití metod
sort(), zápis totožných slov (určených pomocí regulárních výrazů) v
textu
do array a následné spočítání a 
porovnání indexů. Vzhledem k složitějšímu procesu bude zpočátku 
nejvhodnější zkontrolovat, zda neexistuje CPAN modul, který by tento
proces řešil.
\item Výsledná klíčová slova seřazená od nejdůležitějšího bude
nejlepší zapsat do array a pak je po jednom vypsat pomocí cyklu
foreach do terminálu a do textového souboru.
\end{itemize}


\section*{Algoritmizace}

Nejdříve je nutné ověřit, zda je v systému nainstalován CPAN modul
Lingua::EN::Keywords, bez kterého nebude skript fungovat. Program 
tento problém řeší tak, že zavolá externí instalační shell skript 
nacházející se ve stejné složce:

\begin{lstlisting}[label={list:first},caption=Příkaz pro spuštění
instalačního
skriptu --
Analyzátor klíčových slov.]
system("/bin/bash ./instalator_modulu.sh") == 0
  or die "Skript nelze spustit";
\end{lstlisting}

Tento skript otevře CPAN shell a ověří, případně se pokusí
nainstalovat chybějící moduly, a to ideálně bez nutnosti přístupu
root uživatele pomocí modulu local::lib:

\begin{lstlisting}[label={list:first},caption=Instalační skript instalator\_modulu.sh -- Analyzátor klíčových slov.]
#!/usr/bin/sh
eval `perl -I ~/perl5/lib/perl5 -Mlocal::lib`
echo 'eval `perl -I ~/perl5/lib/perl5 -Mlocal::lib`' >> ~/.profile
echo 'export MANPATH=$HOME/perl5/man:$MANPATH' >> ~/.profile
perl -MCPAN -e shell <<END_OF_CPAN_COMMANDS
install local::lib
END_OF_CPAN_COMMANDS
echo "################################"
perl -MCPAN -Mlocal::lib -e 'CPAN::install(LWP)'
echo "################################"
perl -MCPAN -e shell<<END_OF_CPAN_COMMANDS
install Lingua::EN::Keywords
END_OF_CPAN_COMMANDS
echo "################################"
\end{lstlisting}

Po instalaci potřebných modulů se instalační skript
instalator\_modulu.sh ukončí. Skript Klicova\_slova.pl pokračuje
dále, a načte vstupní textový soubor v kódování utf-8.

\begin{lstlisting}[label={list:first},caption=Perl skript načítá vstup -- Analyzátor klíčových slov.]
use Lingua::EN::Keywords;
use utf8;
my $file = "text.txt";
my $document = do {
  local $/ = undef;
  open my $fh, "<:encoding(UTF-8)", $file
  or die "could not open $file: $!";
  <$fh>;
};
\end{lstlisting}

Následuje vyfiltrování importovaného textu od obecných slov pomocí
regulárního výrazu. Jako zdroj českých stop slov posloužil seznam na
webu https://countwordsfree.com/.

\begin{lstlisting}[label={list:first},caption=Odfiltrování stop slov a znaků -- Analyzátor klíčových slov.]
...
$document =~ s/[.]+//gui;
$document =~ s/[?]+//gui;
$document =~ s/[:]+//gui;
$document =~ s/[^A-Za-z]\ba\b//gui;
$document =~ s/[^A-Za-z]\baby\b//gui;
$document =~ s/[^A-Za-z]\bahoj\b//gui;
$document =~ s/[^A-Za-z]\baj\b//gui;
...
\end{lstlisting}

Po vyfiltrování textu definuje skript array @keywords, který je
pomocí metody keywords() modulu Lingua::EN::Keywords naplněn pěti
nejdůležitějšími klíčovými slovy dle frekvence výskytu.

\begin{lstlisting}[label={list:first},caption=Volání metody keywords() modulu Lingua::EN::Keywords -- Analyzátor klíčových slov.]
my @keywords = keywords($document);
\end{lstlisting}

\begin{lstlisting}[label={list:first},caption=Modul Lingua::EN::Tagger extrahuje 5 nejdůležitějších klíčových slov -- Zdroj: https://metacpan.org/source/SIMON/Lingua-EN-Keywords-2.0/Keywords.pm.]
package My::Tagger;
@My::Tagger::ISA=qw(Lingua::EN::Tagger);
my %known_stems;
sub stem {
    my ( $self, $word ) = @_;
    return $word unless $self->{'stem'};
    return $known_stems{ $word } if exists $known_stems{$word};
    my $stemref = Lingua::Stem::En::stem( -words => [ $word ] );
 
    $known_stems{ $word } = $stemref->[0] if exists $stemref->[0];
}
 
sub stems { reverse %known_stems; }
 
package Lingua::EN::Keywords;
use Lingua::EN::Tagger;
require 5.005_62;
use strict;
use warnings;
 
my $t = My::Tagger->new(longest_noun_phrase => 5,weight_noun_phrases=>0);
 
require Exporter;
our @ISA = qw(Exporter);
our @EXPORT = qw( keywords);
our $VERSION = '2.0';
sub keywords {
    my %wl = $t->get_words(shift);
    my %newwl; 
    $newwl{unstem($_)} += $wl{$_} for keys %wl;
    return (sort { $newwl{$b} <=> $newwl{$a} } keys %newwl)[0..5];
}
sub unstem {
    my %cache = $t->stems;
    my $word = shift;
    return $cache{$word} || $word;
}
\end{lstlisting}

V závěru skript spustí dva foreach cykly, které při každé iteraci
vypíší na nový řádek jedno z pěti klíčových slov uložených v
@keywords seřazených dle důležitosti, a to nejdříve do nového
souboru klicovaslova.txt, a poté i do konzole.

\begin{lstlisting}[label={list:first},caption=Výstup skriptu -- Analyzátor klíčových slov.]
foreach (@keywords) {
  binmode(STDOUT, ":utf8");
  print "$_\n";
};

my $filename = 'klicovaslova.txt';
open(my $fh, '>:encoding(UTF-8)', $filename) or die "Could not open file '$filename' $!";

foreach (@keywords) {
  binmode(STDOUT, ":utf8");
  print $fh "$_\n";
};

close $fh;
\end{lstlisting}

Poté se skript ukončí.
\section*{Nedokonalosti a podněty ke zlepšení pro případnou další verzi}
\begin{enumerate}
\item Úsporněji implementovat seznam stop slov.
\item Instalační skript pro CPAN moduly je na některých systémech
nefunkční a je třeba moduly instalovat manuálně.
\item Nahradit činnost modulu vlastním algoritmem, který by mohl
určovat důležitost klíčových slov na základě kritérií zadaných
uživatelem.
\item Odhalit a odstranit příčinu upozornění "Wide character in print..."

\end{enumerate}
\end{document}
