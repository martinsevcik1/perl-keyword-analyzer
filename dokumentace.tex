\batchmode
\documentclass[a4paper,11pt]{article}

\usepackage[T1]{fontenc}
\usepackage[utf8]{inputenc}
\usepackage{graphicx}
\usepackage{xcolor}

\renewcommand\familydefault{\sfdefault}
\usepackage{tgheros}
\usepackage[defaultmono]{droidmono}

\usepackage{amsmath,amssymb,amsthm,textcomp}
\usepackage{enumerate}
\usepackage{multicol}
\usepackage{tikz}

\usepackage{geometry}
\geometry{total={210mm,297mm},
left=25mm,right=25mm,
bindingoffset=0mm, top=20mm,bottom=20mm}


\linespread{1.3}

\newcommand{\linia}{\rule{\linewidth}{0.5pt}}

\newtheoremstyle{mytheor}
    {1ex}{1ex}{\normalfont}{0pt}{\scshape}{.}{1ex}
    {{\thmname{#1 }}{\thmnumber{#2}}{\thmnote{ (#3)}}}

\theoremstyle{mytheor}
\newtheorem{defi}{Definition}

\makeatletter
\renewcommand{\maketitle}{
\begin{center}
\vspace{2ex}
{\huge \textsc{\@title}}
\vspace{1ex}
\\
\linia\\
\@author \hfill \@date
\vspace{4ex}
\end{center}
}
\makeatother


\usepackage{fancyhdr}
\pagestyle{fancy}
\lhead{}
\chead{}
\rhead{}
\lfoot{Analyzátor klíčových slov}
\cfoot{}
\rfoot{Stránka \thepage}
\renewcommand{\headrulewidth}{0pt}
\renewcommand{\footrulewidth}{0pt}


\usepackage{listings}
\lstset{
    language=Perl,
    basicstyle=\ttfamily\small,
    aboveskip={1.0\baselineskip},
    belowskip={1.0\baselineskip},
    columns=fixed,
    extendedchars=true,
    breaklines=true,
    tabsize=4,
    prebreak=\raisebox{0ex}[0ex][0ex]{\ensuremath{\hookleftarrow}},
    frame=lines,
    showtabs=false,
    showspaces=false,
    showstringspaces=false,
    keywordstyle=\color[rgb]{0.627,0.126,0.941},
    commentstyle=\color[rgb]{0.133,0.545,0.133},
    stringstyle=\color[rgb]{01,0,0},
    numbers=left,
    numberstyle=\small,
    stepnumber=1,
    numbersep=10pt,
    captionpos=t,
    escapeinside={\%*}{*)}
}

%%%----------%%%----------%%%----------%%%----------%%%

\begin{document}

\title{Analyzátor klíčových slov}

\author{Bc. Martin Ševčík, Masarykova univerzita}

\date{26/02/2019}

\maketitle

\section*{Popis skriptu}

Skript načte obsah vstupního textového souboru. Poté na základě algoritmů
vyfiltruje z textu tzv. stop slova a provede jeho analýzu. Na základě
analýzy definuje pět klíčových slov, která seřadí sestupně dle frekvence
výskytu v textu. Tato slova vypíše v terminálu a uloží je do nového
textového souboru. Součástí skriptu je i provizorní instalátor, který
nainstaluje potřebné CPAN moduly.

\section*{Vstup a výstup}

Vstupním souborem je \textbf{"text.txt"}, který je potřeba naplnit
libovolným textem určeným k analýze. Výstupem skriptu je \textbf{pět
klíčových slov}, které skript
zobrazí v terminálu, a poté uloží do samostatného textového souboru s názvem
\textbf{"klicovaslova.txt"}.

\section*{Prerekvizity}

Podmínkou pro spuštění skriptu je nainstalavaný Perl verze 5. Skript dále
využívá \textbf{CPAN} modul \textbf{Lingua::EN::Keywords}, jehož funkce je
propojena s modulem
\textbf{Lingua::EN::Tagger} a následně i s \textbf{Lingua::Stem::En.} Tyto
moduly je nutné mít
nainstalované pro správné fungování skriptu.

\section*{Instalace}

Skript obsahuje provizorní automatický instalátor, který ale dle
prozatímního testování nemusí být spolehlivý na všech systémech. Hlavním
kamenem úrazu instalátoru je snaha o obejítí nutnosti instalovat moduly jako
\textit{root} uživatel skrze \textbf{local::lib}. Pokud instalátor selže,
nainstalujte moduly manuálně dle instrukcí na webových stránkách uvedených
níže.

\begin{itemize}
\item https://metacpan.org/source/Lingua::Stem::En
\item https://metacpan.org/source/Lingua::EN::Tagger
\item https://metacpan.org/pod/Lingua::EN::Keywords
\end{itemize}

\section*{Spuštění}

Skript se spustí po otevření složky (pomocí terminálu), ve které se nachází
a zadání následujícího příkazu:




\begin{lstlisting}[label={list:first},caption=Příkaz pro spuštění skriptu --
Analyzátor klíčových slov.]
perl klicova_slova.pl
\end{lstlisting}
Skript je rozdělen do několika dílčích kroků a vyžaduje vstupy od uživatele.
Prvním důležitým vstupem před spuštěním skriptu je soubor \textbf{"text.txt"}, který
se musí nacházet ve stejné složce jako skript a musí být naplněný textem
určeným k analýze. Další vstupy, které skript po spuštění od uživatele
vyžaduje jsou potvrzení průběhu skriptu v jednotlivých popsaných částech
pomocí klávesy \textit{enter}.

\section*{Jak skript probíhá}

\begin{enumerate}
\item Skript ověří přítomnost potřebných CPAN modulů, případně se je pokusí nainstalovat
\item Načte se modul Lingua::EN::Keywords a obsah textového souboru "text.txt"
\item Pomocí regulárních výrazů se z textu odfiltrují stop slova
\item Skript zobrazí filtrovaný dokument
\item Text zobrazí klíčová slova a zapíše je do souboru "klicovaslova.txt"
\item Skript se ukončí
\end{enumerate}
\section*{Dodatek}

Skript pro filtrování stop slov využívá slova ze
zdroje:https://countwordsfree.com/stopwords/czech


\end{document}

